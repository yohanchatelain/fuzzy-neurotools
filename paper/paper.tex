 \documentclass[11pt,onecolumn]{article}
 \usepackage[margin=0.5in]{geometry}

\usepackage[ruled,vlined]{algorithm2e}
\usepackage{amsmath}
\usepackage{xspace}
\usepackage[binary-units=true]{siunitx}
\usepackage{ulem}
%\usepackage{censor}
\usepackage[table]{xcolor}
\usepackage{graphicx}
\usepackage{hyperref}
\usepackage{tabularx}

\usepackage{subcaption}
\usepackage{booktabs}
\usepackage{multirow}
\usepackage{listings}
\usepackage{dingbat}
\usepackage{mathtools}

\hypersetup{
    colorlinks=true,
    linkcolor=black,
    citecolor=blue,
    filecolor=black,
    urlcolor=blue}

\def\BibTeX{{\rm B\kern-.05em{\sc i\kern-.025em b}\kern-.08em
    T\kern-.1667em\lower.7ex\hbox{E}\kern-.125emX}}


\makeatletter
  \def\footnoterule{\kern-3\p@
  \hrule \@width 2in \kern 2.6\p@} % the \hrule is .4pt high
\makeatother


\begin{document}

\newcommand{\fslspm}{FSL-SPM\xspace}
\newcommand{\fslafni}{FSL-AFNI\xspace}
\newcommand{\afnispm}{AFNI-SPM\xspace}
\newcommand{\tristan}[1]{\color{orange}\textbf{From Tristan:} #1\color{black}\xspace}
\newcommand{\ali}[2]{\color{green}\textbf{Ali:} #1\color{black}\xspace}
\newcommand{\discuss}[1]{\uwave{#1}}
\newcommand{\gk}[1]{\color{purple}#1 \textbf{-GK}\color{black}\xspace}


\title{Comparing variability across fMRI analysis software packages and numerical errors}

\author{Ali Salari$^1$, Yohan Chatelain$^1$, Alexander Bowring$^2$, Camille Maumet$^3$, Gregory Kiar$^4$, Tristan Glatard$^1$\\
\vspace*{0.1cm}\\
$^1$ Department of Computer-Science and Software Engineering, Concordia University, Montreal, Canada\\
$^2$ Li Ka Shing Centre for Health Information and Discovery, Nuffield Department of Population Health,\\ Big Data Institute, University of Oxford, Oxford, UK\\
$^3$ Inria, Univ Rennes, CNRS, Inserm, IRISA UMR 6074, Empenn ERL U 1228, Rennes, France\\
$^4$ Center for the Developing Brain, Child Mind Institute, New York, NY, USA}

\maketitle
\begin{abstract}

Variability has been broadly observed in functional MRI analyses across both
differences in analytic tools and numerical instabilities. However,
the relationship between numerical (within-tool) and between-tool variabilities for
given analysis is unclear. We extended a previous comparison
of fMRI analysis software libraries (namely FSL, AFNI, and SPM) and relate the observed differences to the numerical
variability observed using Monte-Carlo arithmetic. We found that
between-tool variability was consistently larger than numerical variability in group-level analyses,
whereas numerical variability approached between-tool variability in some subject-level analyses.
Interestingly, between-tool variability and
machine error appeared moderately correlated ($r > 0.5$) in all tested conditions.
Finally, we found that numerical variability had effects of similar magnitude to between-tool
variability when errors were introduced at the precision of 17 bits --- a
precision level found in commonly-used libraries \gk{What does this mean? They only have precision of 17, or they use data that is 17 bits?}. Our findings motivate the
continued investigation of numerical instability in neuroimaging, and position it 
as a possible proxy for studying arbitrary between-tool variations.
\end{abstract}

\section{Introduction}

Software variability has been highlighted as an important source of error
in fMRI analysis, originating from mainly two factors: between-tool
variability denotes differences in the outcomes of different
software toolboxes --- such as FSL, AFNI and SPM --- when analyzing the same dataset
using similar methods \gk{this sentence says "two factors" but then only mentioned between tool. Also, it is too long}. Numerical variability, in contrast, denotes
differences in outcomes of the same analytical toolbox
executed in different environments – such as different operating systems or
different hardware platforms \gk{you mention these two examples of numerical variability, but you use neither in this paper. Why not mention the one you use?}. This manuscript investigates the extent to
which both types of variability relate to each other.

Numerical variability stems from errors introduced in floating point
computations by numerical rounding, cancellation, and absorption~\cite{muller2018handbook} \gk{only? What about poor conditioning + random effects + noise in data?}. In
practice, the occurrence of such errors is affected by variations in
hardware, software dependencies, compilation, parallelization, and most
likely other factors \gk{what does "most likely other factors" mean?}. In the best case, the impact of such errors and the induced
variations is in the range of 1 unit in the last place (ulp) --- also known
as machine error, the difference between two successive floating-point
numbers. However, when computations depend on imprecise libraries,
numerical error may vary by several ulps. For instance, the numerical error
of the complementary error function (erfc) is larger than 3 ulps in several
mathematical libraries~\cite{zimmermann:hal-03141101}, implying that an update of this function could introduce
numerical perturbations larger than 3 ulps. Due to numerical instability,
machine error may amplify during its propagation through analyses,
resulting in considerable loss of precision in final results. In fMRI analysis,
numerical variability produced by Linux updates has been shown to result in Dice
coefficients ranging from 0.0 to 1.0 for ICA components produced by FSL
MELODIC~\cite{Glatard2015} \gk{there isn't really context for what this result means, yet. Is this a lot? Dice hasn't been explained. Also, how are ICA components (weighted vectors) compared via Dice (a measure for binary image/vector comparison)? }.

Between-tool variability is not a surprising concept: different programs
executed on the same data may obviously produce different results \gk{even when doing conceptually the same thing? Why? Why is this obvious?}. However,
in fMRI analysis, programs all aim to estimate brain activity from measured
variations in the BOLD signal, which is not expected to be measurably
impacted by computational artifacts \gk{Why wouldn't it be? Especially if you state that it isn't surprising for tools to lead to different results, how come you expect fMRI to not suffer from this?}. Therefore, the observation reported
in~\cite{bowring2019exploring} that the three main fMRI analysis toolboxes
may result in marked analytical differences (Dice coefficients ranging from
0.000 to 0.684 for thresholded activation maps), has raised substantial
concern. In a follow-up study~\cite{bowring2021isolating}, the same authors
isolated the analytical components where the pipelines diverge,
highlighting a major impact of the signal model and suggesting that the
root-cause for between-tool variability depends on the analysis design and
task paradigm \gk{If you state this finding, you need to unpack what it means a little bit. What part of analysis design? By task paradigm do you essentially mean the expected effect-size for the task that individuals are performing within the scanner?}. Further, the work in~\cite{Li2021.12.01.470790} found
moderate agreement between results produced by five independent fMRI
analysis pipelines and analyzed in details the components that led to these
variations \gk{This sentence feels inconsistent and counter to the previous, and also doesn't tell us their findings; what components led to variations?}. Between-tool variability is supposedly an important determinant
of the substantial analytical variability recently observed in
fMRI~\cite{botvinik2020variability} \gk{arguably, one of the larger issues that the NARPS result showed us was that people construct and test ``equivalent'' hypotheses differently, so despite highly similar data they will arrive at different conclusions}.

We investigate the relationship between numerical and between-tool
variability through two main questions: (1) what is the relative magnitude of
within-tool (numerical) variability and between-tool variability, and (2) is there an association between within- and between-tool variability. The first question determines the extent to which each variability
source may impact current fMRI results, and which one should be addressed or leveraged
in priority. The second question may provide deeper insights on the origins
of software variability. Conceptually, both types of variability may
reflect a single solution space shaped by local minima and
ill-conditioning \gk{this feels a bit clumsily worded, but I don't immediately have a rewrite that comes to mind }, which could be explored through numerical or software
perturbations.


\section{Materials and Methods}

\subsection{fMRI analysis \& Dataset}

We replicated the analysis described as study `ds000001'
in~\cite{schonberg2012decreasing}, relying on the data publicly available
in OpenNeuro at \url{https://openneuro.org/datasets/ds000001} and using
three widely-used software packages for fMRI data processing, namely FMRIB
Software Library (FSL)~\cite{jenkinson2012fsl}, Analysis of Functional
NeuroImages (AFNI)~\cite{cox1996afni}, and Statistical Parametric
Mapping (SPM)~\cite{penny2011statistical}. We selected this dataset because
comparable analysis pipelines implemented in FSL, AFNI and SPM were already 
publicly available and extensively described in~\cite{bowring2019exploring}.
Furthermore, the work in~\cite{bowring2019exploring} already evaluated the
effect of tool variability for this dataset, which we intended to
extend with the present quantification of numerical variability.

In the selected study, 16 healthy adult subjects participated in the
balloon analog risk task~\cite{lejuez2002evaluation} to measure risk-taking
behavior over three scanning sessions~\cite{schonberg2012decreasing}. We
reused the preprocessing, first-level, and second-level analyses
implemented by~\cite{bowring2019exploring} consistently across all three
software packages. Table~\ref{table:pipeline-steps}, adapted from~\cite{bowring2019exploring},
summarizes the analytical steps in each
pipeline.


%%%%%%%%%% Summary of statstics %%%%%%%%
\setlength{\tabcolsep}{4pt}
\begin{table}[h]
    \centering
    \begin{tabular}{|c|l|c|c|c|}
        \hline
%        \multirow{2}{*}{} & \multicolumn{1}{c}{Thresholded}& & \multicolumn{1}{c}{Unthresholded}& \\
        \multicolumn{2}{|c|}{} & FSL & AFNI & SPM \\
        \hline
        {Preprocessing} & {Motion Correction}                          & \checkmark    & \checkmark     & \checkmark  \\
        {} & {Segmentation}                               &    &      & \checkmark  \\
        {} & {Brain Extraction (Anatomical)}              & \checkmark     & \checkmark    & \checkmark  \\
        {} & {Brain Extraction (Functional)}              &   & \checkmark     &  \\
        {} & {Intra-subject Coregistration}               & \checkmark    & \checkmark     & \checkmark \\
        {} & {Inter-subject Registration}                 & \checkmark    & \checkmark     & \checkmark \\
        {} & {Analysis Voxel Size}                        & \checkmark    & \checkmark     & \checkmark \\
        {} & {Smoothing}                                  & \checkmark    & \checkmark     & \checkmark  \\
        \hline
        {First-level} & {Model Specification}                          & \checkmark    & \checkmark     & \checkmark  \\
        {} & {Inclusion of 6 Motion Parameters}                               & \checkmark   &  \checkmark    & \checkmark  \\
        {} & {Model Estimation}                           & &     & \checkmark  \\
        {} & {Contrasts}                                   &  \checkmark & \checkmark     & \checkmark \\
        \hline
        {Second-level} & {Model Specification}                          & \checkmark    & \checkmark     & \checkmark  \\
        {} & {Model Estimation}                           &      &    & \checkmark  \\
        {} & {Contrasts}                                   &   & \checkmark     & \checkmark  \\
        {} & {Second-level Inference}                               &  \checkmark  &    \checkmark  & \checkmark  \\
        \hline

      \end{tabular}
    \caption{Software processing steps (adapted from~\cite{bowring2019exploring}).}
    \label{table:pipeline-steps}
\end{table}

\subsection{Fuzzy Libmath environment}

To introduce numerical noise in the analyses, we used
Fuzzy Libmath~\cite{salari2021accurate}, a version of the GNU
mathematical library (libmath) instrumented with Monte-Carlo arithmetic.
Monte-Carlo arithmetic simulates numerical errors
errors by introducing a controlled amount of noise in floating-point
operations through the following perturbation~\cite{Parker1997-qq}:

\begin{equation} \label{eq:mca_inexact}
  inexact(x) = x + 2^{e_x-t}\xi,
\end{equation}
where $e_x$ is the exponent in the floating-point representation of $x$,
$t$ is the virtual precision (the number of unperturbed bits in the
mantissa of $x$), and $\xi$ is a random uniform variable of
$(-\frac{1}{2}, \frac{1}{2})$. We introduced the perturbation using 
Verificarlo~\cite{denis2015verificarlo}, an LLVM compiler supporting Monte-Carlo 
arithmetic and other types of numerical instrumentations.

We loaded the instrumented libmath functions in the pipeline using
LD\_PRELOAD, a Linux mechanism to force-load a shared library into an
executable. This mechanism allows functions defined in Fuzzy Libmath to transparently
overload the original ones without the need to modify or recompile the
analysis pipeline.

Fuzzy Libmath introduces numerical perturbations in the values returned by
mathematical functions but not in their input values or within their
implementation. This is done by wrapping the original functions and
applying function \texttt{inexact} to their returned values.
Listing~\ref{algo:wrapper} shows an example of this wrapping for the
\texttt{log} function in single and double precision. In this wrapper, the
original function is called through \texttt{dlsym}, a function that returns
the memory address of a symbol --- in our case \texttt{RTLD\_NEXT}, the
address of the next occurrence of the function in memory. Compiling function wrappers
with Verificarlo instruments the result of the
addition between the original function output and the floating-point zero.

\lstdefinestyle{customasm}{
  belowcaptionskip=1\baselineskip,
  frame=L,
  xleftmargin=\parindent,
  language=[x86masm]Assembler,
  basicstyle=\footnotesize\ttfamily,
  commentstyle=\itshape\color{purple!40!black},
}
\lstinputlisting[caption=Sample wrapper function (C code) \gk{I don't think we need this/it adds anything. Could go to a supplement if we want it in a paper somewhere}, label=algo:wrapper, style=customasm]{wrapper.c}

In~\cite{salari2021accurate}, Fuzzy Libmath was shown to accurately
simulate the effect of Linux operating system updates in structural
pre-processing pipelines of the Human Connectome Project which are largely based on FSL.
 To validate our pipeline instrumentations for the present study, we first verified that non-instrumented
executions of the same pipeline on the same dataset led to identical
results. We also listed the pipeline
library dependencies using the \texttt{ldd} Linux utility and verified that
(1) the tested pipelines were dynamically linked to the GNU libmath library, and 
(2) there was no alternative implementation of elementary mathematical functions in the pipeline dependencies.
Finally, we verified that the use of Fuzzy Libmath affected computational results.

\subsection{Data processing}

We measured between-tool (BT) variability by running the pipelines
described in~\cite{bowring2019exploring} with FSL version 5.0.10, AFNI
version 18.1.09, and SPM12 version r7771 executed with GNU/Octave version
5.2. These software versions were identical to the study
in~\cite{bowring2019exploring} except for SPM for which we used GNU/Octave
instead of MATLAB to enable mathematical function instrumentation using
Fuzzy Libmath. Indeed, MATLAB uses its own built-in mathematical functions,
which prevents the use of Fuzzy Libmath. In AFNI, we set the number of
threads to 7 even though AFNI executions
in~\cite{bowring2019exploring} were single-threaded. This was meant to
reduce the time overhead resulting from Fuzzy Libmath instrumentation 
\gk{Did we verify that single-threading generally didn't lead to different results than 7-threading? As in, were they all ostensibly sampled from the same random distribution?}. 
\ali{We verified that multithreading leads to different results (e.g. single-threading vs, 7-threading). 
However, single-threaded executions couldn't reproduce the original results.}
All the analyses were conducted on the CentOS 7.3 operating system. The
computations were performed on \href{https://www.computecanada.ca}{Compute
Canada's} Béluga cluster nodes, each with 2$\times$ Intel Gold 6148 Skylake~@~2.4~GHz
 (40 cores/node) CPU and 8~GB of RAM per core. To facilitate portability and reproducibility,
 we encapsulated the
above-mentioned software packages in Docker container images based on CentOS 7.3.
Although, we converted Docker containers to Singularity images to enable running on cluster nodes.

We measured numerical (within-tool; WT) variability by running the same analyses three
times using Fuzzy Libmath with a virtual precision of $t=53$~bits for
double-precision values and $t=24$~bits for single-precision values. These
values were chosen such that the numerical perturbation simulates machine
error. The resulting samples were equally plausible estimates of
the true numerical result at the precision used by the pipelines. Moreover, to evaluate numerical perturbations at different magnitudes, 
we repeated the FSL analyses for virtual
precisions ranging from $t=1$~bit to $t=24$~bits for single-precision
and double-precision values. We also repeated the FSL analyses for virtual 
precisions ranging from $t=24$~bits to $t=53$~bits for double-precision values, having 
set the virtual precision to $t=24$~bits for single-precision values. 

We evaluated WT and BT variability for thresholded as well as unthresholded
group-level and subject-level t-statistics maps by computing the
standard deviation of t-statistic maps across pipelines or Fuzzy Libmath samples, respectively. For BT, we computed the standard deviation 
across a given pair of tools.
For WT, we computed the standard deviation across the three
Fuzzy Libmath samples of a given tool. We computed WT variability for a pair of tools (A, B) as follows:
\begin{equation}
  \sigma_{WT(A,B)}^2 = \sigma_{WT(A)}^2 + \sigma_{WT(B)}^2,
  \label{eq:wt-pair}
\end{equation}
where $\sigma_{WT(.)}$ is the numerical variability for a given tool
and $\sigma_{WT(., .)}$ is the numerical variability for a pair of tools.

Further, from the thresholded maps, we determined regional instability
between activation clusters in the 360 regions in the Human Connectome
Project Multi-Modal Parcellation atlas version 1.0
(HCP-MMP1.0)~\cite{glasser2016multi}. For BT, we considered a region
unstable for a pair of tools if it contained activated voxels for a tool
but not for the other one. For WT, we considered a region unstable for a
pair of tools (A, B) if for tool A or tool B it contained activated voxels only for some Fuzzy Libmath
samples.

\section{Results}
The scripts, Docker images, and data to reproduce the results are available
in our GitHub repository at
\url{https://github.com/big-data-lab-team/fuzzy-neurotools}.


\subsection{Validation of Replication}

We verified the correctness of our analyses by comparing our unperturbed
t-statistic group maps with the ones obtained
in~\cite{bowring2019exploring}. For FSL, we found
identical results. For SPM, the files (as measured by checksums) were different but differences
were visually unnoticeable. For AFNI, the activation maps were similar
overall, however, differences were noticeable visually. 
The observed differences remained small (see Supplemental Material~\ref{sec:supp-repro}), and might be due to the use of
GNU/Octave vs MATLAB in SPM, and of multithreading in AFNI. We performed visual quality control of the AFNI
and SPM results for each individual subject and confirmed that T1-weighted images were
correctly skull-stripped and registered to the MNI template.

\subsection{In the group analysis, BT variability was larger than WT variability}

Table~\ref{table:pipeline-stats} presents summary statistics for the
group-level t-statistics. For each tool pair (A, B), BT variability was
significantly larger than machine error in tool A or B, in both thresholded
and unthresholded maps (Wilcoxon signed-rank test and t-test p~\textless~$10^{-5}$ for all tests).
% \gk{we don't have sufficient sample size to say that p is that small,
% so we should probably say less than $10^{-5}$ or something else we could actually
% arrive at with our number of repetitions, pipeline pairs, and samples.}
% for all tests, t-test p=0 \gk{similar comment re p value} for all tests).
Figure~\ref{fig:unthresh-maps}-\textbf{A},\textbf{B},\textbf{C} shows that
these global differences were confirmed regionally. 
In addition, BT and WT variability appeared moderately correlated across voxels (Pearson's R
in [0.56, 0.58], p=0.0, Figure~\ref{fig:unthresh-maps}-\textbf{D}). Interestingly, the correlation 
appears driven by a set of voxels exhibiting comparable BT and
WT values located around the identity line in Figure~\ref{fig:unthresh-maps}D, which, however, did not seem to have 
any spatial consistency.


\setlength{\tabcolsep}{5pt}
\begin{table}[h]
    \centering
    \begin{tabular}{cccccc|cc}
        \toprule
        && \multicolumn{4}{c|}{Group map} & \multicolumn{2}{c}{Subject maps}\\
        \multirow{2}{*}{}& {} & \multicolumn{2}{c}{Thresholded} & \multicolumn{2}{c|}{Unthresholded} & \multicolumn{2}{c}{Unthresholded} \\
        % \cmidrule{3-8}
        {} & {} & $\mu$ & $\sigma$ & $\mu$ & $\sigma$ & $\mu$ & $\sigma$ \\
        \midrule
        \rowcolor{lightgray!50}
        {Between Tools} & FSL vs. SPM        &  1.282       & 0.525      & 0.443     & 0.344  & 0.366       & 0.293     \\
        \rowcolor{lightgray!50}
        {(BT)} & FSL vs. AFNI                &  1.548       & 0.616      & 0.547     & 0.441  & 0.439       & 0.352     \\
        \rowcolor{lightgray!50} 
        {} & AFNI vs. SPM                    &  1.475       & 0.672      & 0.608     & 0.477  & 0.491       & 0.381     \\
        {Within Tool} & FSL                  &  0.354       & 0.491      & 0.082     & 0.065  & 0.077       & 0.054     \\
        {(WT)}   & SPM                       &  0.252       & 0.448      & 0.054     & 0.045  & 0.048       & 0.037     \\
        {}   & AFNI                          &  0.434       & 0.524      & 0.128     & 0.135  & 0.108       & 0.131     \\
        \bottomrule
    \end{tabular}
    \caption{Voxel-wise mean and standard deviation of BT and WT variability
    in t-statistics maps.}
    \label{table:pipeline-stats}
\end{table}

\begin{figure*}[ht]
  \centering
    % \fbox{\begin{minipage}{\dimexpr \textwidth-2\fboxsep-2\fboxrule}
      \begin{subfigure}[ht]{0.9\textwidth}
      \centering
      \includegraphics[width=0.9\textwidth]{figures/std/gl-unthresh.png}
      %\caption{Standard deviation of thresholded t-statistics map on template surface}
      \end{subfigure}
    \caption{Unthresholded group-level variability computed between tools
    (\textbf{A}), within tools at machine error (\textbf{B}), difference
    between them (\textbf{C}), and voxel-wise comparison (\textbf{D}). }
    \label{fig:unthresh-maps}
    % \end{minipage}}
  \end{figure*}

  \subsection{In subject analyses, WT variability approached BT variability in some regions}

  Table~\ref{table:pipeline-stats} also presents summary statistics for
  subject-level unthresholded t-statistics maps. As for group-level maps,
  BT variability was consistently larger than WT variability (Wilcoxon
  signed-rank test and t-test p~\textless~$10^{-5}$ for all tests).
  However, Figure~\ref{fig:unthresh-maps-sbj} shows that for
  some subjects, WT variability approached and even
  surpassed BT in some regions. As for the group maps, BT and WT
  variability appeared moderately correlated for all subjects (R in [0.442,
  0.612], p=0.0 for all subjects, see Supplemental Material~\ref{sec:supp-subjects}).

  %%%%%%%%%% Var. of Unthresh sbj05%%%%%%%%
\begin{figure*}[ht]
    \centering
    \includegraphics[width=0.9\textwidth]{figures/std/sbj05-std.png}
    %\caption{Standard deviation of thresholded t-statistics map on template surface}
    \caption{For subject with highest WT variability,
    unthresholded subject-level variability computed between tools (\textbf{A}), within tools at machine error (\textbf{B}),
     and difference between them (\textbf{C}).}
  \label{fig:unthresh-maps-sbj}
\end{figure*}

\subsection{At precision t=17~bits, WT variability approached BT variability in the group analysis}

While the previous results were obtained for machine error, we also
evaluated numerical variability across different virtual precisions for FSL
and found that the virtual precision of t=17 bits minimized the RMSE
between BT and WT in unthresholded group maps.
FSL produced group maps with $\mu=0.325$ and $\sigma=0.287$ at this virtual precision,
which indicates a bit lower variability compared to BT variability
(Wilcoxon signed-rank test and t-test p~\textless~$10^{-5}$ for all tests).
Figure~\ref{fig:gnp-mni} shows BT
and WT in the corresponding group maps obtained at t=17 bits, showing that
BT and WT reach comparable magnitudes in some regions. BT and WT remained
moderately correlated at this precision. 

  
  %%%%%%%%%% plot different precisions%%%%%%%%
  \begin{figure*}[ht]
      \begin{subfigure}[ht]{\textwidth}
        \centering
        \includegraphics[width=.75\textwidth]{figures/bg_global_precision.pdf}
        %\caption{Standard deviation of thresholded t-statistics map on template surface}
        \end{subfigure}
        %\caption{Standard deviation of thresholded t-statistics map on template surface}
        \caption{Unthresholded group t-statistics standard deviations computed between tools (\textbf{A}),
          within tools at the virtual precision of t=17~bits (\textbf{B}), difference between them (\textbf{C}), and 
          voxel-wise comparison (\textbf{D}).}
    \label{fig:gnp-mni}
  \end{figure*}


  \subsection{Previous results were confirmed in thresholded group maps}

  Figure~\ref{fig:thresh-maps}-\textbf{A},\textbf{B},\textbf{C} compares BT
  and WT for thresholded group maps. Thresholding is an unstable operation
  that introduced variability at the edges of
  active regions for both BT and WT. Except in these regions, BT remained consistently larger
  than WT \gk{what regions? This kind of came from nowhere}. Moreover, to measure correlation between BT and WT, we
  measured WT instability and BT instability in each region of the
  HCP-MMP1.0 parcellation \gk{again, it feels kind of abrupt and I'm not really sure what this section is getting at or how it differs from the previous based on these descriptions. Perhaps introduce that thresholding is common, but since it makes things discontinuous we study it through aparcellation, etc...}. The confusion matrices in
  Figure~\ref{fig:thresh-maps}-\textbf{D} report these instabilities for
  the 360 tested regions. The average ratio of unstable regions was 26\%
  for BT and 9.2\% for WT, which confirmed that BT variability was larger than machine error.
  The average Cohen's kappa score\footnote{$\kappa \leq 0$ denotes chance agreement, $-1 \leq \kappa \leq 1$}
   between WT instability and BT instability was $\kappa=0.17$, indicating 
  a moderate agreement between WT instability and BT instability.

  %%%%%%%%%% Var. of Thresh %%%%%%%%
  \begin{figure*}[ht]
        \begin{subfigure}[ht]{\textwidth}
          \centering
          \includegraphics[width=\textwidth]{figures/std/gl-thresh.png}
          %\caption{Standard deviation of thresholded t-statistics map on template surface}
          \end{subfigure}
        \centering
      \caption{Thresholded group t-statistics standard deviations computed between tools (\textbf{A}),
      within tools at the virtual precision of t=17~bits (\textbf{B}), difference between them (\textbf{C}), and 
      confusion matrices of activation instability
      in BT and WT among the 360 regions of the HCP-MMP1.0 parcellation (\textbf{D}).}
      \label{fig:thresh-maps}
    \end{figure*}
  

\section{Discussion}

In fMRI group analyses, within-tool variability remains an order of magnitude smaller
than between-tool variability. Group analyses benefit from regularization
of numerical noise, which is expected to increase as sample size
increases. Therefore, for fMRI studies with large sample sizes, within-tool variability may be negligible with respect to between-tool variability. The
recommendation made in~\cite{botvinik2020variability} to rely on
``multiverse'' analyses where multiple analysis tools are compared is
therefore likely to successfully correct for machine error in group
studies. Nevertheless, within-tool variability remains substantial in group analyses
that are based on a single tool, as is commonly the case in current fMRI
studies. In particular, in our study, the inherent instability of
thresholding was triggered by machine error in 9\% of 360 brain regions,
which indicates that machine error might have impacted neuroscientific
conclusions related to these regions. 

In subject-level analyses, within-tool and between-tool variabilty can become
of comparable magnitude for some subjects in some regions. This observation
is particularly relevant to the development of fMRI-based biomarkers aiming
at individualized phenotype predictions. Machine error is expected to play
a non-negligible role in such analyses, even when predictions combine
results produced by multiple tools. The observed regularization effect of
group analyses toward numerical noise is consistent with observations made
in~\cite{kiar2020numerical} from diffusion MRI where connectome graph
statistics were found to be substantially unstable at the subject-level
while group distributions remained consistent. \gk{this last sentence feels like it belongs in the previous paragraph, right before mentioning thresholding}

For both group- and subject-level analyses, between-tool variability and within-tool
variability were found to be moderately correlated. Even though between-tool
variability and within-tool  variability are different in nature, this result
suggests that in some cases both sources of variability may have a common
cause that might be related to the conditioning of the BOLD signal
estimation problem \gk{I wouldn't necessarily say it's because of "BOLD signal estimation problem", just "algorithms used to process"} in a specific dataset. Therefore, in some regions,
instabilities of similar magnitude may be triggered by small numerical
perturbations, model variations, or implementation differences. For
instance, the analysis of datasets with high motion may be unstable both 
across tools and numerically. 

Therefore, numerical stability may be a suitable proxy to study
between-tool variability. This speculation might be of practical value to
address software variability at large given that numerical stability refers
to a consistent mathematical framework whereas between-tool variability
remains more empirical. Likewise, useful quality control metrics may be derived from numerical and between-tool stability.

The finding that numerical variability approached tool variability at the
virtual precision of t=17~bits is interesting too. Indeed, while machine
error generally introduces differences in the order of 1 ulp --- or
t=53~bits for double-precision values and t=24~bits for single-precision
ones --- common scientific software dependencies introduce larger errors.
For instance, SciPy's 2D spline interpolation was recently found to be
precise up to t=10~bits~\cite{pytracer} \tristan{Fix ref} and might therefore introduce
numerical perturbations leading to errors in the range of between-tool
variability \tristan{Yohan, please check that sentence}. Such high errors
might be triggered by updates in operating systems, Python, MATLAB, and
other software dependencies.

Our results are limited by the type of numerical noise introduced in the
analyses. We only perturbed the outputs of elementary
mathematical functions while numerical noise could creep in any
floating-point operation. Our estimation of numerical variability should
therefore be considered a lower bound. Our estimation of tool varability is
also likely to be underestimated, having tested only 3 analytical pipelines
among the thousands available~\cite{carp2012plurality}.

In conclusion, our results motivate further numerical stability
investigations in fMRI analysis pipelines. Pipeline-level analyses could be
conducted to identify specific components that contribute to numerical
variability, and if possible correct them accordingly. When instability is
inherent to the analysis, sampling results distributions through numerical
perturbations might improve stability, as explored in~\cite{kiar2021data}.
Finally, pipeline-specific statistical corrections might be envisaged to
account for numerical variability.

\bibliographystyle{plain}
\bibliography{biblio}

\clearpage

\setcounter{equation}{0}
\setcounter{figure}{0}
\setcounter{table}{0}
\setcounter{section}{0}

\makeatletter
\renewcommand{\theequation}{S\arabic{equation}}
\renewcommand{\thefigure}{S\arabic{figure}}
\renewcommand{\thesection}{S\arabic{section}}

\textbf{\centering \Large Supplemental Materials}

\section{Reproduced results}
\label{sec:supp-repro}

Figure~\ref{fig:replication-diff} shows the difference in SPM and AFNI
group analyses maps between the results in~\cite{bowring2019exploring} and
our replication. Numerical perturbations of 1 ulp are likely to have been
introduced by our replication due to the use of GNU/Octave vs MATLAB for SPM
and multithreading for AFNI. However, the group maps remained very similar
overall, which led us to conclude that our results correctly reproduced 
the ones in~\cite{bowring2019exploring}.
\begin{figure*}[ht]
  % \fbox{\begin{minipage}{\dimexpr \textwidth-2\fboxsep-2\fboxrule}
  % \centering
  \includegraphics[width=\textwidth]{figures/replication-diffs.png}
  \caption{Differences between reproduced and original results obtained in~\cite{bowring2019exploring} 
  of unthresholded group-level t-statistics for SPM (left) and AFNI (right).}
  \label{fig:replication-diff}
% \end{minipage}}
\end{figure*}

\section{BT and WT correlations for all subjects}
\label{sec:supp-subjects}

Figure~\ref{fig:unthresh-correlation-allsbj} plots the relationship between
WT and BT for each subject, showing a consistent moderate correlation between BT and
WT.

\begin{figure*}[ht]
      \centering
      \includegraphics[width=.6\textwidth]{figures/sbj-std-corr-unthresh-plot.png}
      \caption{Comparison between BT and WT variabilty for 16 subjects.}
  \label{fig:unthresh-correlation-allsbj}
\end{figure*}

\end{document}
\endinput
